\documentclass[]{article}

% Mathematics packages
\usepackage{amsmath, amsfonts, amssymb, dsfont}
\usepackage{float}
\usepackage{fancyhdr}
\usepackage{fontawesome5}

\usepackage{arydshln} 
\usepackage{tikz}
\usetikzlibrary{automata, positioning}

\usepackage[utf8]{inputenc}

\usepackage{animate}


% Graphics and Figures
\usepackage{graphicx, adjustbox, float, xcolor}

% TikZ and PGFPLOTS
\usepackage{tikz, pgfplots}
\pgfplotsset{compat=1.17}

\usetikzlibrary{arrows.meta, automata, positioning, decorations.pathreplacing, decorations.markings}
\allowdisplaybreaks[2]

% Formatting
\usepackage{booktabs, enumitem, caption}

\newcommand{\sgn}{\text{sgn}}

\pagestyle{fancy}
\setlength{\headheight}{14.5pt} % Ensure enough space for header
\setlength{\textheight}{24.5cm} % Ensures body text fits well

\setlength{\topmargin}{-2cm} % Moves everything up

\lhead{Ben Vickers - 304024}
\rhead{Mathematical Biology Assignment 1}
\setlength{\headsep}{0.4cm} % Adds a gap between header and body text



\title{\textbf{Mathematical Biology Assignment 1}}
\author{Ben Vickers - 304024}
\date{\textbf{Due: 7:59 PM, Thursday 16 October 2025}}

\begin{document}
	
	\maketitle
\noindent \textbf{Question}

\vspace{1em}

The growth rate of a fish species at time $t$ is described by the equation
\[
\frac{dg}{dt} = c g^{1/2}(t) - a g(t),
\]
\indent where $c$ and $a$ are positive constants. The initial condition is $g(0) = g_0$ (the \indent initial size), where $g_0$ is a positive constant.\newline

\noindent \textbf{(a)} Without solving the differential equation, determine lim$_{t\to \infty}\, g(t) $.\newline

\noindent \textbf{SOLUTION:}\newline

To determine lim$_{t\to \infty}\, g(t) $, we can find the stable steady state(s) of $g(t)$.

\[
\frac{dg}{dt} = c g^{1/2}(t) - a g(t) \implies f(g) =c g^{1/2} - a g 
\]

Let $g^*$ be a steady state, then we have $f(g^*) = 0 \iff c {g^*}^{1/2} - a g^* =0 $ 

Solving for $g^*$ we have:

\[
{g^*}^{1/2} \left(c - a {g^*}^{1/2}\right) = 0
\]

Which gives two roots:

\[
\begin{aligned}
\displaystyle &{g^*_1}^{1/2} = 0 &&\implies \displaystyle g^*_1 = 0 \\
\displaystyle &c - a {g^*_2}^{1/2} = 0 \implies \displaystyle {g^*_2}^{1/2} = \frac{c}{a} &&\implies \displaystyle g^*_2 = \left(\frac{c}{a}\right)^2
\end{aligned}
\]

We have determined there are two steady states: $g^*_1 = 0\, , \, g^*_2 = \left(\frac{c}{a}\right)^2$. \newline

To determine lim$_{t\to \infty}\, g(t) $, we check which of these steady states is stable.\newline

Notice:
\[
f(g) =c g^{1/2} - a g  \implies f'(g) = \frac{c}{2g^{1/2}} - a
\]

However, attempting to evaluate $f'(g)$ at $g^*_1$ yields:

\[
f'(g^*_1) = \left[{\frac{c}{2\cdot \sqrt{g^*_1}} - a}\right]_{g^*_1 = 0} = \frac{c}{0} - a
\]

As this is clearly not defined, we can employ a graphical approach to evaluate \indent the eigenvalue of $g^*_1$ from its right-sided limit:

\begin{figure}[H]
    \centering
    \includegraphics[width=0.75\linewidth]{f(g).png}
    \caption{$f(g) =c g^{1/2} - a g  $; a=2 \, c=4}
    \label{fig:f(g)}
\end{figure}
Inspecting the gradient of $f(g)$ about both steady states, we can clearly \indent notice that:

\[
\left. \frac{df(g)}{dg} \right|_{g=0^{+}} > 0 \, \text{ and } \, \left. \frac{df(g)}{dg} \right|_{g=\frac{c^2}{a^2}} < 0
\]

Specifically:

\[
\left. \frac{df(g)}{dg} \right|_{g=\frac{c^2}{a^2}} = \left[{\frac{c}{2\cdot \sqrt{g^*_2}} - a}\right]_{g^*_2 = \frac{c^2}{a^2}} = \frac{c}{2 \cdot \frac{c}{a} } - a = \frac{a}{2} -\frac{2a}{2} = - \frac{a}{2}  < 0.
\]

Therefore, we can conclude that $g^*_1 = 0$ is an unstable steady state, and \indent $g^*_2 = \frac{c^2}{a^2}$ is a stable steady state.\newline

I.e. because we have shown that $g^*_2 = \frac{c^2}{a^2}$ is a (and the only) stable steady \indent state of the system, we have:
\[
\text{lim}_{t\to \infty}\, g(t) = \frac{c^2}{a^2} := g^* \quad \Box
\]

Notice that since we are given $a, c \, > 0 \implies g^* > 0$ and therefore, our \indent solution is biologically valid. \newline

\noindent \textbf{(b)} Now solve the differential equation explicitly to find
\[
g(t) = \frac{1}{a^2} \left( c - \big(c - a g_0^{1/2}\big) e^{-at/2} \right)^2.
\]

\noindent \textbf{SOLUTION:}\newline

As hinted, we begin by making the substitution $g(t) = h^2(t)$ which yields:

\[
\frac{dg(t)}{dt} = 2h(t)\frac{dh(t)}{dt}
\]

So we have:

\[
\frac{dg(t)}{dt} =  c g^{1/2}(t) - a g(t) = c h(t) -ah^2(t) = 2h(t)\frac{dh(t)}{dt}
\]

As we observed in Figure 1 for $f(g) = g'$:
\[
\begin{cases}
    f(g) > 0 \quad ;&g\in(0,g^*)\\
    f(g) < 0  \quad ;& g \in (g^*,\infty)
\end{cases} \implies  \begin{cases}
    g' > 0 \implies g(t) \in [g_0,g^*] \quad ;& 0<g_0 < g^* \\
    g ' < 0 \implies g(t) \in [g^* , g_0] \quad ;& 0<g^*<g_0
\end{cases}: t\geq 0 \]
\[\implies g(t) \geq \min\{g_0,g^*\} > 0 \, \forall t \geq 0  \implies g(t) \neq 0 \implies h(t) \neq 0.
\]

%Notice that since $g'(t) > 0 \mid g(t) \in (0,\frac{c^2}{a^2}) \, , \, g'(t) < 0 \mid g(t) \in (\frac{c^2}{a^2},\infty) \implies $%
%Since $f(g)$ has unique roots at $g=0, g=\frac{c^2}{a^2}, \, f$, must either be strictly increasing or decreasing on $g\in (0,\frac{c^2}{a^2}) \cup(\frac{c^2}{a^2},\infty)$  
%$a,c,g_0 > 0 \implies  g(t) \geq \min\{g_0,g^*\}>0$%

%$g(t) \neq 0 \implies h(t) \neq 0$:%
Therefore, we can divide through by $h(t)$.
\[
c h(t) -ah^2(t) = 2h(t)\frac{dh(t)}{dt} \implies c - ah(t) = 2\frac{dh(t)}{dt}
\]

Applying separation of variables, we have:

\[
dt = -2 \frac{dh(t)}{ah(t)-c}
\]

Integrating this expression then yields:

\[
\int{dt} = -2\int{\frac{dh(t)}{ah(t)-c}}
\]

\[
\implies t = -\frac{2}{a} \, \text{ln}\left|ah(t)-c\right| + \text{ln(k)}
\]

\[
\implies -\frac{at}{2} = \text{ln}\left|ah(t)-c\right| + \text{ln(k)}
\]

\[
\implies e^{-\frac{at}{2}} = k\cdot (ah(t)-c)
\]

Substituting $g(t) = h^2(t)\implies h(t) = \sqrt{g(t)}$ into our current expression:

\[
e^{-\frac{at}{2}} = k\cdot (a \sqrt{g(t)}-c)
\]

Now we can apply initial conditions ($t=0$) to find $k$ by evaluating:

\[
\left. e^{-\frac{at}{2}} \right|_{t=0} = \left.  k\cdot (a \sqrt{g(t)}-c) \right|_{t=0}
\]

\[
\implies 1 = k \cdot (a \sqrt{g_0}-c)
\]

\[
\implies k = \frac{1}{a \sqrt{g_0}-c}
\]

Substituting this value of $k$ back into our expression yields:

\[
e^{-\frac{at}{2}} = \left(\frac{1}{a \sqrt{g_0}-c}\right)\cdot (a \sqrt{g(t)}-c)
\]

Now, solving for $g(t)$:

\[
\left(a \sqrt{g_0}-c\right)e^{-\frac{at}{2}} = a \sqrt{g(t)}-c
\]

\[
\implies \left(c - a \sqrt{g_0}\right)e^{-\frac{at}{2}} = c - a \sqrt{g(t)}
\]

\[
\implies c - \left(c - a \sqrt{g_0}\right)e^{-\frac{at}{2}} = a \sqrt{g(t)}
\]

\[
\implies \frac{1}{a}\left(c - \left(c - a \sqrt{g_0}\right)e^{-\frac{at}{2}}\right) =  \sqrt{g(t)}
\]

\[
\implies \frac{1}{a^2}\left(c - \left(c - a \sqrt{g_0}\right)e^{-\frac{at}{2}}\right)^2 = g(t) \quad \Box
\]

This matches the form of the proposed solution to this differential equation.\newline

We can also notice that since our solution is in the form of a product \indent of squares, this means $g(t) > 0 \, \forall t \in [0,\infty)$ which confirms the solutions' \indent biological validity.\newline


\noindent \textbf{(c)} Verify that the equilibrium solution $g = \tfrac{c^2}{a^2}$ satisfies the differential equation.\newline

\noindent \textbf{SOLUTION:}\newline

Evaluating (LHS), if $g(t) = g = \tfrac{c^2}{a^2}$, then we have:

\[
\frac{dg}{dt} = \frac{d}{dt}\left( \frac{c^2}{a^2}\right) = 0
\]

Evaluating (RHS), if $g(t) = g = \tfrac{c^2}{a^2}$, then we have:
\[
 g^{1/2}(t) = \left(\frac{c^2}{a^2}\right)^{1/2} = \frac{c}{a}
\]
\[
\implies c g^{1/2}(t) - a g(t)= c \left(\frac{c}{a}\right) - a\left( \frac{c^2}{a^2}\right)
\]
\[
= \frac{c^2}{a} - \frac{c^2}{a} = 0
\]

I.e. (LHS) = 0 = (RHS) confirming that the equilibrium solution $g = \tfrac{c^2}{a^2}$ \indent satisfies the differential equation. $\quad \Box$ \newline


\noindent \textbf{(d)} Determine the value of $g(t)$ when $t = 0$ using the initial condition.\newline

\noindent \textbf{SOLUTION:}\newline

Applying our obtained solution to this system, and evaluating at $t=0$, we \indent have:
\[
\left. g(t)\right|_{t=0} = \left. \frac{1}{a^2}\left(c - \left(c - a \sqrt{g_0}\right)e^{-\frac{at}{2}}\right)^2 \right|_{t=0}
\]

\[
\implies g(0) = \frac{1}{a^2}\left(c - \left(c - a \sqrt{g_0}\right)e^{-\frac{a\cdot 0}{2}}\right)^2 
\]
\[
= \frac{1}{a^2}\left(c - \left(c - a \sqrt{g_0}\right)\right)^2 =\frac{1}{a^2} \left(a \sqrt{g_0}\right)^2  =\frac{1}{a^2} \cdot a^2 \cdot g_0 = g_0 
\]

I.e. we have confirmed that our solution for $g(t)$ respects the given initial \indent condition: $\left.g(t)\right|_{t=0} = g(0) = g_0.\quad \Box $ \newline

\noindent \textbf{(e)} Find the time $t$ when the population size $g(t)$ reaches half of its equilibrium value.\newline

\noindent \textbf{SOLUTION:}\newline

We aim to solve the following equation for $t$:
\[
g(t) = \frac{1}{2} \cdot \frac{c^2}{a^2}
\]

We will need to be careful about how we approach this to ensure we don't \indent get any biologically invalid solutions.\newline

We can start with this stage of our working from (b):

\[
\left(a \sqrt{g_0}-c\right)e^{-\frac{at}{2}} = a \sqrt{g(t)}-c \to \left(a \sqrt{g_0}-c\right)e^{-\frac{at}{2}} = a \sqrt{\frac{g^*}{2}}-c
\]
\[
\implies \left(a \sqrt{g_0}-c\right)e^{-\frac{at}{2}} = a \sqrt{\frac{c^2}{2a^2}}-c
\]

We can now solve this for $t$:
\[ \left(a \sqrt{g_0}-c\right)e^{-\frac{at}{2}} = \frac{c}{\sqrt{2}}-c
\]

\[
\implies e^{-\frac{at}{2}} = \frac{\frac{c}{\sqrt{2}}-c}{ \left(a \sqrt{g_0}-c\right)} = \frac{c-\frac{c}{\sqrt{2}}}{ \left(c-a \sqrt{g_0}\right)}
\]

To ensure a valid solution, we need to notice that $e^{-\frac{at}{2}} > 0 \, \forall \, t$. Therefore, \indent we must have:

\[
\frac{c-\frac{c}{\sqrt{2}}}{ \left(c-a \sqrt{g_0}\right)} = \frac{c(1-\frac{1}{\sqrt{2}})}{ \left(c-a \sqrt{g_0}\right)}> 0
\]

Since $c > 0\, , 1-\frac{1}{\sqrt{2}}>0 \implies c(1-\frac{1}{\sqrt{2}})>0$. So our condition requires:
\[
\left(c-a \sqrt{g_0}\right) > 0 \iff g_0 < g^*
\]

Next, because we only consider $t\in[0,\infty)$ to model a biologically valid \indent system, on this domain we will have $0 <e^{-\frac{at}{2}} \leq 1$.\newline

The lower bound has been ensured by the condition $g_0 < g^*$. \newline

Now we check what conditions must be imposed to ensure $e^{-\frac{at}{2}} \leq 1$:

\[
e^{-\frac{at}{2}}  = \frac{c-\frac{c}{\sqrt{2}}}{ \left(c-a \sqrt{g_0}\right)} \leq 1
\]

Because we have conditioned $g_0 < g^* \iff \left(c-a \sqrt{g_0}\right) > 0 $, our inequality \indent is unaffected by multiplication through by the denominator.

\[
\implies c-\frac{c}{\sqrt{2}} \leq c-a \sqrt{g_0}
\]
\[
\implies \frac{c}{\sqrt{2}} \geq a \sqrt{g_0} \implies \frac{c^2}{2a^2} = \frac{g^*}{2}\geq g_0 
\]

So we have $g_0 \in \left(0,\frac{g^*}{2}\right]$ as our condition to ensure the solution, $t$ remains \indent biologically valid.\newline

Note the following:\begin{enumerate}
    \item If $g_0 = \frac{g^*}{2}$ then $t=0$ (this will be a result of the following form of $t$, and is trivial.
    \item If $g_0 \in (\frac{g^*}{2}, g^*)$ then whilst the solution for $t$ will exist, it requires $t<0$ which is biologically invalid.
    \item If $g_0 \in \left[g^*, \infty\right)$ then $g > g^* \, \forall t \implies \not\exists \, t \in \left[ 0,\infty\right) $ such that $g(t) = \frac{g^*}{2}$. 
\end{enumerate}

We can now proceed to solve for $t$ given our imposed conditions:

\[
e^{-\frac{at}{2}}  = \frac{c(1-\frac{1}{\sqrt{2}})}{ \left(c-a \sqrt{g_0}\right)}
\]
\[
\implies e^{-\frac{at}{2}}=\frac{1 -  \frac{1}{\sqrt{2}}}{ \left(1 - \frac{a}{c} \sqrt{g_0}\right)}
\]

\[
\implies   -\frac{at}{2}= \text{ln}\left( \frac{1 -  \frac{1}{\sqrt{2}}}{ \left(1 - \frac{a}{c} \sqrt{g_0}\right)} \right)
\]

\[
\implies t = \frac{2}{a}\,\text{ln}\left( \frac{1 - \frac{a}{c} \sqrt{g_0}}{1 -  \frac{1}{\sqrt{2}}} \right)\quad;\, g_0 \in \left(0,\frac{g^*}{2}\right] \quad \Box
\]

Figure 2 demonstrates the accuracy of this solution.

\begin{figure}[H]
    \centering
    \includegraphics[width=0.75\linewidth]{ass1e.png}
    \caption{Intersection of $g(t)$ and half of the equilibrium value; $a=2 ,\, c=4$}
    \label{fig:f(g)}
\end{figure}

\noindent \textbf{(f)} Discuss the effect of increasing the parameter $a$ on the long-term behaviour of $g(t)$.\newline

\noindent \textbf{SOLUTION:}\newline

\noindent \textbf{1)} Decreased Equilibrium \newline

The first effect of increasing $a$ on the long-term behaviour of $g(t)$ is a \indent decrease in the equilibrium value. \newline

We have shown that the stable steady state / equilibrium of $g(t) = \frac{c^2}{a^2} = \indent g^*>0$. \newline

Therefore, if $a \to ma \, , \, m>0$, then:
\[g^* = \frac{c^2}{a^2} \to \frac{c^2}{(ma)^2} =\frac{1}{m^2} \left(\frac{c^2}{a^2}\right) = \frac{1}{m^2} \cdot g^* \]

I.e. increasing $a$ by a factor of $m$ will decrease the equilibrium by a factor \indent of $m^2$.\newline

\noindent \textbf{2)} Decreased recovery time. \newline

The second effect of increasing $a$ on the long-term behaviour of $g(t)$ is the \indent system will experience a decrease in the recovery time.\newline

As we saw previously, 
\[
f'(g^{*})=-\frac{a}{2} \implies T_R\left(\frac{c^2}{a^2} \right) = \mathcal{O}\left(\frac{1}{f'(g^*)} \right) =  \mathcal{O}\left(-\frac{2}{a} \right) = \frac{2}{a}
\]

So, as $a$ increases, the Recovery Time will decrease, inversely to $a$.\newline

\noindent \textbf{3)} Increased equilibrium convergence rate. \newline

By evaluating:
\begin{align*}
|\displaystyle g(t)-g^{*}|
&=\left|\frac{1}{a^{2}}\Big(c-\big(c-a\sqrt{g_{0}}\big)e^{-\frac{a}{2}t}\Big)^{2}-\frac{c^{2}}{a^{2}} \right|\\[2pt]
&=\left|\frac{1}{a^{2}}\left[-2c\big(c-a\sqrt{g_{0}}\big)e^{-\frac{a}{2}t}+\big(c-a\sqrt{g_{0}}\big)^{2}e^{-a t}\right]\right| \\[2pt]
& \leq \frac{1}{a^{2}}\left|\left[-2c\big(c-a\sqrt{g_{0}}\big)e^{-\frac{a}{2}t} \right] \right| + \frac{1}{a^{2}}\left| \left[\big(c-a\sqrt{g_{0}}\big)^{2}e^{-a t}\right]\right| \\[2pt]
&= \frac{2c\left|c-a\sqrt{g_{0}}\right|}{a^{2}}e^{-\frac{a}{2}t} + \frac{\big(c-a\sqrt{g_{0}}\big)^{2}}{a^{2}}\left( e^{-a t}\right) \\[2pt]
& \leq \frac{2c\left|c-a\sqrt{g_{0}}\right|}{a^{2}}e^{-\frac{a}{2}t} + \frac{\big(c-a\sqrt{g_{0}}\big)^{2}}{a^{2}}\left(e^{-\frac{a}{2}t}\right) \\[2pt]
& = \frac{2c\left|c-a\sqrt{g_{0}}\right| + \big(c-a\sqrt{g_{0}}\big)^{2}}{a^{2}}e^{-\frac{a}{2}t} \\[2pt]
& =C_a(c,g_0) e^{-\frac{a}{2}t} \to 0.
\end{align*}
\indent we can see that $g^*$ is exponentially stable with convergence rate $\frac{a}{2}$ and \indent therefore, by increasing $a$, the rate at which $g(t)$ converges to $g^*$ increases.\newline
\linebreak
\noindent\textbf{4)} Change of trajectory direction towards $g^*$.\newline

As we observed in (b):
\[
\left(a \sqrt{g_0}-c\right)e^{-\frac{at}{2}} = \left(a \sqrt{g(t)}-c\right)
\]

So we have that:
\[
g'(t) = \sqrt{g(t)}\left(c-a\sqrt{g(t)}\right) = -\sqrt{g(t)}\left(a\sqrt{g(t)}-c\right) 
\]
\[= -\underbrace{\sqrt{g(t)}}_{>0}\left(\left(a \sqrt{g_0}-c\right)\underbrace{e^{-\frac{at}{2}}}_{>0} \right)\]

That is to say:
\[
\sgn\left(g'(t)\right) = -\sgn\left(a-\sqrt{g_0} c\right) = \begin{cases}
    1 & ;\, a < \frac{c}{\sqrt{g_0}} \\
    -1 & ;\, a > \frac{c}{\sqrt{g_0}}
\end{cases} \quad : \forall \, t \in [0,\infty)
\]

Meaning $g(t)$ is monotone on $[0,\infty)$ and given fixed values for $c$ and $g_0$, $g$ \indent will approach $g^* = \frac{c^2}{a^2}$ from either above, or below depending on $a$. I.e. by \indent increasing $a$, $g(t)$ will eventually become a strictly decreasing function that \indent approaches $g^*$ from above.\newline

\noindent \textbf{Conclusion}\newline
{\par\leftskip=\parindent \parindent=0pt \rightskip=0pt plus 1fil
We have shown in f$(1,2,3,4)$ that as the parameter $a$ increases, $g(t)$ will approach an increasingly smaller equilibrium value, $g^* = \frac{c^2}{a^2}$ at an increasingly faster rate with decreasing recovery time, and, once $a$ increases beyond a certain point ($\frac{c}{\sqrt{g_0}}$), $g(t)$ will become a decreasing function, approaching $g^*$ from above. $\quad \Box$\par}

\hrulefill 



\end{document}


